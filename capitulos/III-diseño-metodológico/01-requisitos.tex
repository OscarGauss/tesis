\section{REQUISITOS}

\subsection{REQUISITOS FUNCIONALES}

Los requisitos funcionales se desarrollaron en una matriz de trazabilidad como se muestra en la Tabla 1. ya que así se permitirá monitorear cada uno de los requisitos durante el ciclo de vida del proyecto para asegurar que se están cumpliendo de manera eficaz.

 \begin{longtable}[c]{| m{0.5cm} | m{3.5cm} | m{3.5cm} | m{3cm} | m{3.5cm} |}
    \hline
    Id & Descripción & Criterios de aceptación & Objetivo & Desarrollo o Planificación
    \\ \hline
    \endfirsthead
    \endhead
    1
    & El sistema debe tener registro de usuario y login del mismo. 
    & Usuario apto para crear cuenta e iniciar en ella dentro de la aplicación web. Inicialmente cada usuario tiene un rol de “user logged”. El usuario administrador (admin) será responsable de asignar privilegios.
    & Personalizar la aplicación web para todo tipo de usuario
    & Este requisito deberá inicializar desarrollo MERN de la aplicación web
    \\ \hline
    2
    & El sistema deberá tener una página principal
    & Al momento de ingresar al sistema el usuario deberá estar en la página de inicio de forma automática.
    & El objetivo de crear una página inicial será de guiar al usuario al uso del sistema.
    & Esta página de inicio deberá estar contenido de noticias en general, de información sobre concurso y de blogs que hayan posteado algunos usuarios.
    \\ \hline
            3
            & El sistema deberá
proveer
un área de tutoriales
            & Los tutoriales deberán abarcar
varios temas respectivos a
concursos de programación. Por
ejemplo, tutoriales
sobre diferentes algoritmos,
ejercicios propuestos y
soluciones
            & El objetivo de tener
tutoriales en el sistema será
para el
aprendizaje de los usuarios
interesados en los concursos
de
programación.
            & Será necesario que el sistema
tenga una sección de tutoriales
donde al ingresar a uno en
particular se deberá llevar a otra
página con el tutorial en texto plano.
            \\ \hline
            4
            & El sistema deberá tener
un motor de
evaluación.
            & El sistema deberá evaluar
los programas presentados por
los usuarios.
            & El objetivo principal de todo
el sistema en sí es la
evaluación automática de
los programas presentados
por los usuarios.
            & El motor de búsqueda será la
parte más importante en el sistema,
será necesario implementar un motor
ya existente que esté correctamente
comprobado en su éxito respecto a la
evaluación.
            \\ \hline
            5
            & El sistema deberá tener
una lista de problemas
propuestos.
            & El sistema deberá desplegar una
lista con el nombre y puntuación
de cada problema.
    & El objetivo de desplegar una
lista de problemas en el
sistema es para que el
usuario pueda identificar y
resolver los problemas.
    & Se deberá desplegar una nueva
página proveniente de cada subcategoría, la cual presentará una lista
de problemas.
Será importante que el sistema tenga
la opción de importar archivos
comprimidos para la adición de
nuevos problemas planteados por
usuarios administradores.
    \\ \hline
\end{longtable}

\subsection{REQUISITOS NO FUNCIONALES}

Luego del análisis del sistema proyectado se identificaron los siguientes requerimientos no
funcionales:
\begin{itemize}
    \item No será posible estar autenticado en más de una máquina.
    \item La contraseña de usuario deberá tener un mínimo de 6 caracteres.
    \item Los avatares en el perfil de usuario se irán desbloqueando de acuerdo a su nivel y puntaje. 
\end{itemize}

\textbf{Seguridad}
\begin{itemize}
    \item El sistema utilizará un protocolo SSL para realizar conexiones seguras entre clientes.
    \item El sistema registrará al usuario con una contraseña segura JWT.
\end{itemize}

\textbf{Confiabilidad}
\begin{itemize}
    \item El sistema permitirá almacenar datos de manera correcta y completa en la base de datos, además de ello ofrece seguridad y confidencialidad.
\end{itemize}

\textbf{Portabilidad}
\begin{itemize}
    \item El sistema será portable ya que deberá ser un sistema web que debe funcionar únicamente en la web y no requiere de instalación, sino de una conexión activa a internet.
    \item El sistema deberá ser portable en cualquier navegador.
\end{itemize}

\textbf{Disponibilidad}
\begin{itemize}
    \item El sistema deberá estar disponible en un 100\% para todo el público, cuando sea el momento de realizar un mantenimiento el sistema mostrará un mensaje y es en ese momento que dejará de estar disponible para el público la cantidad de tiempo que sea requerido.
\end{itemize}

\textbf{Mantenibilidad}
\begin{itemize}
    \item Se deberá realizar un estudio que determine cuál será el tiempo necesario para realizar un mantenimiento necesario al sistema.
    \item Se deberán realizar cambios y actualizaciones necesarias cada vez que el programador vea por necesario, estas se deberán realizar a parte del tiempo de mantenibilidad.
\end{itemize}

\section{DIAGRAMA DE CASOS DE USO}

\section{DIAGRAMA DE SECUENCIA}
