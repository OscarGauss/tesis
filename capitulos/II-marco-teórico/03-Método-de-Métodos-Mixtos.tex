\section{MÉTODO DE MÉTODOS MIXTOS}

El enfoque de métodos mixtos (MM) es un diseño de investigación que integra tanto métodos cuantitativos como cualitativos en un mismo estudio o programa de investigación \autocite{Creswell2018}. Esta integración busca aprovechar las fortalezas de ambos enfoques para obtener una comprensión más completa y profunda del fenómeno en estudio.

Los MM han ganado popularidad en las últimas décadas debido a su capacidad para abordar preguntas de investigación complejas que no pueden ser respondidas adecuadamente con un solo enfoque \autocite{Johnson2017}. Además, los MM ofrecen una mayor flexibilidad y adaptabilidad a diferentes contextos y situaciones de investigación.

\subsection{COMPONENTES CLAVE}

Los MM se caracterizan por tres componentes clave:

\begin{enumerate}
    \item \textbf{Recolección de datos:} Los MM implican la recolección de datos tanto cuantitativos (numéricos) como cualitativos (textuales, visuales, etc.). Esta recolección puede ser simultánea o secuencial, dependiendo del diseño de investigación elegido.
    \item \textbf{Análisis de datos:} Los MM requieren la aplicación de técnicas de análisis tanto cuantitativas (estadísticas) como cualitativas (análisis de contenido, teoría fundamentada, etc.). El análisis de datos mixtos puede ser complejo, pero ofrece una rica variedad de perspectivas y hallazgos.
    \item \textbf{Integración de datos:} La integración de datos es un proceso clave en los MM, que implica combinar los resultados cuantitativos y cualitativos para generar una interpretación coherente y completa del fenómeno en estudio. Esta integración puede ocurrir en diferentes etapas del proceso de investigación, desde la formulación de preguntas de investigación hasta la presentación de los resultados.
\end{enumerate}

\subsection{JUSTIFICACIÓN DEL USO DE MÉTODOS MIXTOS}

La justificación del uso de MM se basa en varios argumentos:

\begin{itemize}
    \item \textbf{Triangulación:} Los MM permiten la triangulación de datos, es decir, la verificación de los resultados obtenidos con un método mediante el uso de otro método. Esta triangulación aumenta la validez y confiabilidad de los hallazgos \autocite{Jick1979}.
    \item \textbf{Complementariedad:} Los datos cuantitativos y cualitativos se complementan entre sí, proporcionando una visión más holística y multifacética del fenómeno en estudio \autocite{Greene2007}.
    \item \textbf{Desarrollo de teoría:} Los MM facilitan el desarrollo de teorías más completas y matizadas, al integrar diferentes perspectivas y niveles de análisis \autocite{Morse1991}.
    \item \textbf{Flexibilidad:} Los MM ofrecen una gran flexibilidad en el diseño de la investigación, permitiendo a los investigadores adaptar sus métodos a las necesidades específicas de cada estudio \autocite{Johnson2017}.
\end{itemize}

\subsection{DISEÑOS DE MÉTODOS MIXTOS}

Existen diversos diseños mixtos, cada uno con sus propias características y propósitos:

\begin{itemize}
    \item \textbf{Diseño convergente paralelo:} Los datos cuantitativos y cualitativos se recolectan y analizan simultáneamente, y los resultados se comparan y contrastan \autocite{Creswell2018}.
    \item \textbf{Diseño secuencial explicativo:} Primero se recolectan y analizan datos cuantitativos, y luego se recolectan datos cualitativos para explicar o profundizar en los hallazgos cuantitativos \autocite{Creswell2018}.
    \item \textbf{Diseño secuencial exploratorio:} Primero se recolectan y analizan datos cualitativos para generar hipótesis o preguntas de investigación, y luego se recolectan datos cuantitativos para probar o refinar dichas hipótesis \autocite{Creswell2018}.
    \item \textbf{Diseño anidado concurrente:} Un tipo de datos (cuantitativo o cualitativo) se incrusta dentro del otro para responder a diferentes preguntas de investigación \autocite{Creswell2018}.
\end{itemize}

\subsection{PROCESO DE INVESTIGACIÓN}

El proceso de investigación con MM sigue los siguientes pasos:

\begin{enumerate}
    \item \textbf{Formulación de preguntas de investigación:} Las preguntas de investigación deben ser claras, relevantes y adecuadas para ser abordadas con un enfoque mixto.
    \item \textbf{Diseño de la investigación:} El diseño de investigación debe especificar los métodos cuantitativos y cualitativos que se utilizarán, así como la secuencia y la forma en que se integrarán los datos.
    \item \textbf{Recolección de datos:} La recolección de datos debe ser rigurosa y sistemática, siguiendo los procedimientos establecidos para cada método.
    \item \textbf{Análisis de datos:} El análisis de datos debe ser apropiado para cada tipo de datos, utilizando técnicas cuantitativas y cualitativas.
    \item \textbf{Integración de datos:} La integración de datos debe ser cuidadosa y reflexiva, buscando identificar patrones, relaciones y contradicciones entre los resultados cuantitativos y cualitativos.
    \item \textbf{Interpretación de los resultados:} La interpretación de los resultados debe ser coherente con los objetivos de la investigación y con el marco teórico utilizado.
    \item \textbf{Presentación de los resultados:} Los resultados deben ser presentados de forma clara, concisa y accesible para el público objetivo.
\end{enumerate}

\subsection{APLICACIONES EN LA INVESTIGACIÓN}

Los MM se han aplicado en una amplia variedad de campos de investigación, como la educación, la salud, la psicología, la sociología y la administración de empresas. Algunos ejemplos de aplicaciones incluyen:

\begin{itemize}
    \item \textbf{Evaluación de programas:} Los MM pueden utilizarse para evaluar la eficacia de programas sociales, educativos o de salud, combinando datos cuantitativos sobre los resultados del programa con datos cualitativos sobre las experiencias de los participantes.
    \item \textbf{Estudios de caso:} Los MM pueden utilizarse para realizar estudios de caso en profundidad, combinando datos cuantitativos sobre el contexto del caso con datos cualitativos sobre las perspectivas de los actores involucrados.
    \item \textbf{Investigación acción participativa:} Los MM pueden utilizarse en la investigación acción participativa para involucrar a los participantes en el diseño y la implementación de la investigación, combinando datos cuantitativos sobre los resultados de la acción con datos cualitativos sobre las percepciones y experiencias de los participantes.
\end{itemize}

\subsection{VENTAJAS Y DESVENTAJAS}

Como ventajas tenemos:

\begin{itemize}
    \item \textbf{Triangulación:} Aumenta la validez y confiabilidad de los hallazgos.
    \item \textbf{Complementariedad:} Proporciona una visión más holística y multifacética del fenómeno en estudio.
    \item \textbf{Desarrollo de teoría:} Facilita el desarrollo de teorías más completas y matizadas.
    \item \textbf{Flexibilidad:} Permite adaptar los métodos a las necesidades específicas de cada estudio.
\end{itemize}

Como desventajas tenemos:
\begin{itemize}
    \item \textbf{Complejidad:} Requiere una mayor planificación y coordinación.
    \item \textbf{Recursos:} Puede requerir más tiempo, recursos financieros y personal capacitado.
    \item \textbf{Análisis de datos:} Puede ser desafiante integrar diferentes tipos de datos y aplicar diferentes técnicas de análisis.
\end{itemize}

\subsection{CONCLUSIÓN}

El método de métodos mixtos es un enfoque de investigación valioso y versátil que ofrece una serie de ventajas para abordar preguntas de investigación complejas. Sin embargo, también presenta desafíos que deben ser considerados cuidadosamente al diseñar e implementar un estudio mixto.
