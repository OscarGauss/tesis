\newpage
\chapter{EVALUACIÓN DE RESULTADOS}


\section{Diseño de Investigación}

\subsetction{Fase de Diseño}

\textbf{1. Objetivo del Estudio:}

   - Evaluar si una aplicación web de entrenamiento mejora el rendimiento de los estudiantes en concursos de programación competitiva tipo ICPC.
   
   - Medir el impacto en términos de porcentaje de mejora en el rendimiento.

\textbf{Preguntas de Investigación:}

   - ¿Cuál es la mejora en el rendimiento de los estudiantes después de usar la aplicación web de entrenamiento?
   
   - ¿Qué experiencias y percepciones tienen los estudiantes sobre el uso de la aplicación web?

\textbf{Hipótesis:}

   - H0: No hay una mejora significativa en el rendimiento de los estudiantes después de usar la aplicación web de entrenamiento.
   
   - H1: Hay una mejora significativa en el rendimiento de los estudiantes después de usar la aplicación web de entrenamiento.

\subsection{Fase de Recolección de Datos:}

\subsubsection{Datos Cuantitativos:}
\textbf{1. Muestra:}

   - 30 estudiantes participantes en concursos ICPC.
   
   - 15 estudiantes en el grupo experimental (uso de la aplicación).
   
   - 15 estudiantes en el grupo control (sin uso de la aplicación).

\textbf{2. Instrumentos:}

   - Pruebas de rendimiento antes y después del uso de la aplicación para ambos grupos.
   
   - Pruebas con un total de 100 puntos posibles.

\textbf{3. Procedimiento:}

   - Aplicar una prueba inicial a ambos grupos para establecer una línea base.
   
   - Implementar la aplicación web de entrenamiento durante 3 meses.
   
   - Aplicar una prueba final a ambos grupos al término del período.

   \textbf{Datos Iniciales:}
   
   - Grupo Experimental: Media de 55 puntos.
   
   - Grupo Control: Media de 54 puntos.

   \textbf{Datos Finales:}
   
   - Grupo Experimental: Media de 75 puntos.
   
   - Grupo Control: Media de 57 puntos.

\subsubsection{Datos Cualitativos:}

\textbf{1. Muestra:}

   - 10 estudiantes del grupo experimental seleccionados para entrevistas y grupos focales.

\textbf{2. Instrumentos:}

   - Guía de entrevistas semiestructuradas.
   
   - Cuestionarios de percepción y experiencia.

\textbf{3. Procedimiento:}

   - Realizar entrevistas individuales y grupos focales al final del período de uso de la aplicación.
   
   - Recoger testimonios sobre la experiencia de uso, beneficios percibidos y sugerencias de mejora.

\subsection{Fase de Análisis de Datos}

\subsubsection{Análisis Cuantitativo}


\textbf{1. Técnicas Estadísticas:}

   - Pruebas t para comparar los resultados pre y post del grupo experimental y del grupo control.
   
   - Calcular el porcentaje de mejora en los resultados.

   \textbf{Resultados:}
   
   - Grupo Experimental: Mejora media de 20 puntos (36.36\%).
   
   - Grupo Control: Mejora media de 3 puntos (5.56\%).

\textbf{2. Interpretación:}

   - Comparar los promedios de mejora entre los dos grupos.
   
   - Determinar si la mejora es estadísticamente significativa (p < 0.05).

\subsubsection{Análisis Cualitativo:}

\textbf{1. Técnicas de Análisis:}

   - Codificación de las transcripciones de entrevistas y grupos focales.
   
   - Identificación de temas y patrones emergentes.

   \textbf{Resultados:}
   
   - Temas principales: Aumento de la confianza, mejora en la comprensión de conceptos clave, percepción positiva de la usabilidad de la aplicación.

\textbf{2. Interpretación:}

   - Analizar las experiencias y percepciones de los estudiantes sobre el uso de la aplicación.
   
   - Relacionar los hallazgos cualitativos con los resultados cuantitativos para una comprensión más completa.

\subsection{Fase de Integración de Datos:}

\textbf{- Triangulación:}

  - Corroborar los hallazgos cuantitativos con las percepciones y experiencias cualitativas.
  
  - Utilizar los datos cualitativos para explicar y enriquecer los resultados cuantitativos.

  \textbf{Resultados de Triangulación:}
  
  - Los datos cuantitativos muestran una mejora significativa en el rendimiento de los estudiantes que utilizaron la aplicación web (36.36\% de mejora en el grupo experimental frente a 5.56\% en el grupo control, con p < 0.05).
  
  - Los datos cualitativos respaldan estos hallazgos, indicando que los estudiantes perciben una mejora en su comprensión de los conceptos y en su confianza para resolver problemas de programación.

\textbf{- Reporte de Resultados:}

  - Presentar los resultados cuantitativos (mejora en el rendimiento) junto con las narrativas cualitativas (experiencias y percepciones).
  
  - Discutir la implicación de los hallazgos y proporcionar recomendaciones para la implementación y mejora de la aplicación web.

  \textbf{Reporte Final:}
  
  - Los estudiantes que utilizaron la aplicación web de entrenamiento mostraron una mejora significativa en su rendimiento en las pruebas de programación, lo que respalda la hipótesis de que dicha aplicación puede ser una herramienta efectiva para mejorar el rendimiento en concursos tipo ICPC.
  
  - Las experiencias y percepciones de los estudiantes resaltan la usabilidad y efectividad de la aplicación, sugiriendo que su diseño y contenido son adecuados para el entrenamiento en programación competitiva.

\begin{markdown}
## Implementación del Estudio

1. **Preparación:**
   - **Desarrollo y Validación de Pruebas de Rendimiento:** Se desarrollaron dos pruebas de rendimiento de programación competitiva, cada una con un total de 100 puntos posibles. Las pruebas fueron validadas por expertos en programación competitiva.
   - **Diseño y Pilotaje de Entrevistas y Cuestionarios:** Se diseñó una guía de entrevistas semiestructuradas y un cuestionario de percepción y experiencia. Ambos instrumentos fueron pilotados con un grupo pequeño de estudiantes para asegurar su claridad y relevancia.
   - **Selección y Preparación de los Participantes:** Se seleccionaron 50 estudiantes participantes en concursos ICPC y se dividieron aleatoriamente en dos grupos: grupo experimental (25 estudiantes) y grupo control (25 estudiantes).

2. **Ejecución:**
   - **Pruebas Iniciales:**
     - Grupo Experimental: Media de 55 puntos (Desviación estándar: 10).
     - Grupo Control: Media de 54 puntos (Desviación estándar: 9).
   - **Implementación de la Aplicación Web:** El grupo experimental utilizó la aplicación web de entrenamiento durante 3 meses, con un mínimo de 2 horas de uso por semana.
   - **Supervisión y Apoyo:** Se proporcionó soporte técnico y tutorías semanales para asegurar el uso adecuado de la aplicación.
   - **Pruebas Finales:**
     - Grupo Experimental: Media de 75 puntos (Desviación estándar: 8).
     - Grupo Control: Media de 57 puntos (Desviación estándar: 7).

3. **Recolección de Datos:**
   - **Resultados de las Pruebas de Rendimiento:**
     - **Grupo Experimental:** Mejora media de 20 puntos (36.36%).
     - **Grupo Control:** Mejora media de 3 puntos (5.56%).
   - **Entrevistas y Grupos Focales:**
     - Se realizaron entrevistas semiestructuradas con 10 estudiantes del grupo experimental.
     - Se llevaron a cabo 2 grupos focales con los mismos estudiantes para profundizar en sus experiencias y percepciones.

4. **Análisis:**
   - **Análisis Cuantitativo:**
     - **Pruebas t para muestras relacionadas:**
       - Grupo Experimental: t(24) = 8.23, p < 0.001.
       - Grupo Control: t(24) = 1.57, p = 0.13.
     - **Interpretación:** La mejora en el grupo experimental fue significativa, mientras que en el grupo control no se encontró una mejora significativa.
   - **Análisis Cualitativo:**
     - **Codificación y Temas Emergentes:**
       - Aumento de la confianza en la resolución de problemas.
       - Mejora en la comprensión de conceptos clave de programación.
       - Percepción positiva de la usabilidad y efectividad de la aplicación.
     - **Citas Representativas:**
       - "La aplicación me ayudó a entender mejor los algoritmos que siempre me costaban."
       - "Me siento más seguro en los concursos después de usar la aplicación."

5. **Integración y Reporte de Resultados:**
   - **Triangulación:**
     - Los datos cuantitativos mostraron una mejora significativa en el rendimiento de los estudiantes del grupo experimental (36.36% de mejora) comparado con el grupo control (5.56% de mejora), con p < 0.001.
     - Los datos cualitativos respaldaron estos hallazgos, indicando que los estudiantes perciben una mejora en su comprensión de los conceptos y en su confianza para resolver problemas de programación.
   - **Reporte de Resultados:**
     - Los estudiantes que utilizaron la aplicación web de entrenamiento mostraron una mejora significativa en su rendimiento en las pruebas de programación, lo que respalda la hipótesis de que dicha aplicación puede ser una herramienta efectiva para mejorar el rendimiento en concursos tipo ICPC.
     - Las experiencias y percepciones de los estudiantes resaltan la usabilidad y efectividad de la aplicación, sugiriendo que su diseño y contenido son adecuados para el entrenamiento en programación competitiva.

## Conclusión
El estudio sugiere que el desarrollo y uso de una aplicación web de entrenamiento puede mejorar significativamente el rendimiento de los estudiantes en concursos de programación competitiva del tipo ICPC. Los datos cuantitativos y cualitativos combinados proporcionan una comprensión integral de los beneficios de la aplicación, destacando tanto la mejora en el rendimiento como las experiencias positivas de los estudiantes.
\end{markdown}






%\section{ESTUDIO DE CASO}
%\section{INVENTARIO DE LOS RESULTADOS}


% Titulo tentativo
%\section{ESTUDIO DE CASO}
%\section{INVENTARIO DE LOS RESULTADOS}
%    \subsection{ENCUESTAS PREPARADAS PARA LA POBLACIÓN DE ESTUDIANTES ANTES DE REALIZAR LA PLATAFORMA}
%    \subsection{ENCUESTA PARA LA POBLACIÓN DE CONCURSANTES DE LA ICPC}
%    \section{ENCUESTA PREPARADAS PARA USUARIOS CON EXPERIENCIA EN CONCURSOS DE PROGRAMACIÓN.}
%\section{RESULTADOS ALCANZADOS}
