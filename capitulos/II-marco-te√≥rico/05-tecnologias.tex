\section{APLICACIONES WEB}

Las aplicaciones web han emergido como un componente esencial de la infraestructura digital moderna. Estas aplicaciones, accesibles a través de navegadores web, han revolucionado la forma en que interactuamos con la información, los servicios y entre nosotros. Desde plataformas de comercio electrónico y redes sociales hasta herramientas de productividad y aplicaciones empresariales, las aplicaciones web desempeñan un papel fundamental en nuestra vida cotidiana. \citep{flanagan2011javascript}

\subsection{CARACTERÍSTICAS DE LAS APLICACIONES WEB MODERNAS}

Las aplicaciones web modernas se caracterizan por una serie de características clave que las distinguen de las aplicaciones web tradicionales:

\begin{itemize}
    \item \textbf{Interactividad:} Las aplicaciones web modernas ofrecen una experiencia de usuario altamente interactiva, similar a la de las aplicaciones de escritorio. Esto se logra mediante el uso de tecnologías como JavaScript, HTML5 y CSS3, que permiten crear interfaces de usuario dinámicas y responsivas.
    \item \textbf{Acceso a Datos en Tiempo Real:} Muchas aplicaciones web modernas se conectan a bases de datos en tiempo real para proporcionar información actualizada al instante. Esto es especialmente importante en aplicaciones como redes sociales, plataformas de trading y sistemas de monitoreo.
    \item \textbf{Escalabilidad:} Las aplicaciones web modernas están diseñadas para escalar fácilmente a medida que aumenta la demanda. Esto se logra mediante el uso de arquitecturas distribuidas y tecnologías de escalamiento horizontal, como contenedores y microservicios.
    \item \textbf{Seguridad:} La seguridad es una preocupación crítica en el desarrollo de aplicaciones web. Las aplicaciones web modernas implementan medidas de seguridad sólidas para proteger los datos de los usuarios y prevenir ataques maliciosos.
\end{itemize}

\subsection{FRONTEND Y BACKEND}

Es crucial comprender los conceptos de frontend y backend en el contexto del desarrollo web.

\subsubsection{FRONTEND}
El frontend, también conocido como "lado del cliente", se refiere a la parte de una aplicación web con la que los usuarios interactúan directamente. Incluye todo lo que el usuario ve y experimenta en su navegador, como la interfaz de usuario, los elementos visuales, la navegación y las interacciones. El desarrollo del frontend se centra en crear una experiencia de usuario atractiva, intuitiva y eficiente. \citep{basham2018developing}

\subsubsection{BACKEND}
El backend, o "lado del servidor", es la parte de una aplicación web que se ejecuta en el servidor y maneja la lógica de negocio, el acceso a datos y otras operaciones que no son visibles para el usuario. El backend se encarga de procesar las solicitudes del frontend, interactuar con bases de datos, realizar cálculos y enviar respuestas al frontend. \citep{rahman2017learning}

\subsection{STACK MERN}

El desarrollo de aplicaciones web ha experimentado una transformación radical en los últimos años, impulsada por la creciente demanda de experiencias de usuario dinámicas, interactivas y eficientes. En este contexto, el stack MERN (MongoDB, Express, React, Node.js) ha emergido como una solución poderosa y versátil para abordar estos desafíos.

\subsubsection{MONGODB}
MongoDB, una base de datos NoSQL de código abierto, se destaca por su modelo de datos flexible y escalable. A diferencia de las bases de datos relacionales tradicionales, MongoDB utiliza un modelo basado en documentos, lo que permite almacenar datos en un formato similar a JSON (BSON). Esta característica facilita la gestión de datos estructurados, semiestructurados y no estructurados, brindando una mayor agilidad en el desarrollo de aplicaciones. \citep{choquet2018mongodb}

MongoDB se caracteriza por:
\begin{itemize}
    \item \textbf{Modelo de Datos Flexible:} Utiliza un modelo de datos basado en documentos (formato BSON, similar a JSON), permitiendo almacenar datos estructurados, semiestructurados y no estructurados de manera eficiente.
    \item \textbf{Escalabilidad Horizontal:} MongoDB está diseñado para escalar horizontalmente, lo que significa que se pueden agregar más servidores a medida que crece la demanda de la aplicación.
    \item \textbf{Consultas Poderosas:} Ofrece un lenguaje de consulta rico y flexible que permite realizar búsquedas complejas y agregaciones de datos.
\end{itemize}

\subsubsection{EXPRESS}
Express, un framework web minimalista y flexible para Node.js, se ha convertido en una herramienta esencial para el desarrollo de APIs y aplicaciones web del lado del servidor. Su arquitectura modular y su amplia gama de middleware permiten a los desarrolladores construir aplicaciones personalizadas y escalables de manera eficiente. \citep{holmes2014pro}

Express simplifica el desarrollo de backend con:
\begin{itemize}
    \item \textbf{Rutas y Middleware:} Express simplifica la definición de rutas (endpoints) de la API y el uso de middleware para manejar tareas como autenticación, autorización y registro.
    \item \textbf{Minimalista y Flexible:} Express es un framework minimalista que brinda una base sólida para construir APIs personalizadas y escalables.
\end{itemize}

\subsubsection{NODE.JS}
Node.js, un entorno de ejecución de JavaScript basado en eventos, ha revolucionado el desarrollo web al permitir utilizar JavaScript tanto en el frontend como en el backend. Esta capacidad de utilizar un único lenguaje en todo el stack de desarrollo simplifica el proceso de desarrollo y fomenta la reutilización de código y habilidades. \citep{tilkov2016node}

Node.js ofrece ventajas como:
\begin{itemize}
    \item \textbf{JavaScript en el Backend:} Node.js permite utilizar JavaScript tanto en el frontend como en el backend, lo que facilita la reutilización de código y habilidades.
    \item \textbf{Eficiencia y Rendimiento:} Su arquitectura basada en eventos y no bloqueante lo hace ideal para aplicaciones en tiempo real y de alta concurrencia.
\end{itemize}

\subsubsection{NEXT.JS}
React es una libreria de JavaScript declarativa, eficiente y flexible para construir interfaces de usuario. Le permite componer interfaces de usuario complejas a partir de fragmentos de código pequeños y aislados llamados "componentes".

Next.js, un framework de React, ha ganado popularidad por su enfoque en el renderizado del lado del servidor (SSR) y su capacidad para crear aplicaciones web optimizadas para SEO y rendimiento. Next.js ofrece una serie de características que simplifican el desarrollo de aplicaciones React, como enrutamiento automático, generación de páginas estáticas y optimización de imágenes. \citep{vercel2023next}

Next.js aporta al frontend:
\begin{itemize}
    \item \textbf{Renderizado del Lado del Servidor (SSR):} Next.js ofrece SSR, mejorando el rendimiento y el SEO en comparación con el renderizado tradicional del lado del cliente (CSR).
    \item \textbf{Rutas y Enrutamiento:} Simplifica la creación de rutas y el manejo de navegación en aplicaciones React.
    \item \textbf{Optimización de Imágenes y Código:} Next.js incluye herramientas para optimizar imágenes y generar código eficiente, mejorando la velocidad de carga de la aplicación.
\end{itemize}