\section{PROBLEMA}

\subsection{ANTECEDENTES}

Existen varias iniciativas para apoyar a las personas en la programación de aprendizaje. Por ejemplo, los concursos de programación son una buena motivación para que las personas aprendan y mejoren sus habilidades.
Sin embargo, los concursantes deben aprender por sí mismos si no se organizan entrenamientos. Los concursos como las olimpiadas también apuntan a la excelencia, excluyendo a las personas menos calificadas. Un aspecto positivo de los concursos son las discusiones que desencadena entre los concursantes, especialmente para los concursos no en línea \citep{Combéfis-Saint_Marcq}.

\subsubsection{PLATAFORMAS WEB DE PROGRAMACIÓN}

\begin{itemize}
    \item visualgo: VisuAlgo fue conceptualizado en 2011 por el Dr. Steven Halim como una herramienta para ayudar a sus estudiantes a comprender mejor las estructuras de datos y los algoritmos, permitiéndoles aprender los conceptos básicos por su cuenta y a su propio ritmo. VisuAlgo contiene muchos algoritmos avanzados que se analizan en el libro del Dr. Steven Halim (Programación competitiva, en coautoría con su hermano, el Dr. Félix Halim) y más allá. Inicialmente fue diseñado para estudiantes de la Universidad Nacional de Singapur (NUS) que toman varias clases de estructura de datos y algoritmos (por ejemplo, CS1010, CS1020, CS2010, CS2020, CS3230 y CS3230), esta plataforma fue adoptaba por la comunidad de programación competitiva. VisuAlgo no está diseñado para funcionar bien en pantallas táctiles pequeñas (por ejemplo, teléfonos inteligentes) desde el principio debido a la necesidad de satisfacer muchas visualizaciones de algoritmos complejos que requieren muchos píxeles y gestos de hacer clic y arrastrar para interactuar. La resolución mínima de pantalla para una experiencia de usuario respetable es 1024x768 y solo la página de destino es relativamente amigable para dispositivos móviles \citep{visualgo}.
    \item algoexpert.io: Es una plataforma web para prepararse en programación y algoritmia, si bien en esta plataforma se puede encontrar material de muy buena calidad sobre algoritmia esta no esta enfocada a programación competitiva si no a pruebas de entrevistas de trabajo. Para acceder a los recursos que te ofrece la plataforma se debe contar con una suscripción anual de paga que puede ir desde los 60 dolares americanos hasta los 160 dolares americanos por año.
    \item Coursera: Coursera fue fundada por Daphne Koller y Andrew Ng con la visión de proporcionar experiencias de aprendizaje transformadoras para cualquier persona, en cualquier lugar. Ahora es una plataforma de aprendizaje en línea para la educación superior, donde 66 millones de estudiantes de todo el mundo vienen a aprender habilidades del futuro \citep{coursera}. 
    \item platzi: John Freddy Vega y Christian Van Der Henst describen a Platzi como educación online efectiva, Platzi es una de las primeras plataformas que tiene contenidos de tecnología en español enfocado para personas de Latinoamerica. Si bien todo los recursos están en español no se encuentra material especializado en programación competitiva.
\end{itemize}

\subsubsection{TRABAJOS SIMILARES}

\begin{itemize}
    \item [\textbullet]
    \begin{itemize}[nosep]
        \item[] \textbf{Título:} Método conectivo bajo presión v-bloom para el aprendizaje de programación competitiva orientada a participantes de la olimpiada boliviana de informática 
        \item[] \textbf{Autor:} Quispe Valdez, Saúl
        \item[] \textbf{Institución:} Universidad Mayor de San Andrés
        \item[] \textbf{Gestión:} 2017
        \item[] \textbf{Descripción:} Esta investigación trabaja sobre el tema de tecnología informática en el ámbito educativo. Inicialmente se presenta un sustento teórico para luego diseñar el denominado método conectivo bajo presión V-Bloom, orientado a la enseñanza aprendizaje de programación competitiva para estudiantes de nivel escolar con edades comprendidas 11 y 15 años. En una segunda instancia se prueba la validez del mismo en 3 evaluaciones correspondientes a las 4 etapas de la 7ma Olimpiada Científica Estudiantil Plurinacional Boliviana 2017 en el área de Informática (Nivel 2). Finalmente, se proporciona una interpretación de los resultados obtenidos y se presentan las conclusiones del trabajo.
    \end{itemize}

    \item [\textbullet]
    \begin{itemize}[nosep]
        \item[] \textbf{Título:} Tutor inteligente para el proceso de aprendizaje en algoritmos y programación en el lenguaje java
        \item[] \textbf{Autor:} Espejo Cuba, Jose Gonzalo
        \item[] \textbf{Institución:} Universidad Mayor de San Andrés
        \item[] \textbf{Gestión:} 2015
        \item[] \textbf{Descripción:} El presente trabajo aborda la formación de estudiantes principiantes en el área de algoritmos y programación en el lenguaje Java desarrollando un tutor inteligente que mejora el proceso de aprendizaje de algoritmos y programación en el lenguaje Java siendo este uno de los tantos lenguajes de programación, elegido por ser portable, gratuito, se pueden desarrollar programas incluso para celulares y cada día se desarrollan nuevas herramientas.
    \end{itemize}
\end{itemize}