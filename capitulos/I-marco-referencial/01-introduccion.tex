\section{INTRODUCCIÓN}

Los concursantes que participan en las competencias de programación competitiva son principalmente alumnos con experiencia en programación autodidacta que vienen preparándose con recursos en su totalidad o mayoría en ingles.

% La programación de computadoras es el proceso de desarrollar e implementar varios conjuntos de instrucciones para permitir que una computadora realice una determinada tarea, resuelva problemas y proporcione interactividad humana. Estas instrucciones (códigos fuente que están escritos en un lenguaje de programación) se consideran programas de computadora y ayudan a que la computadora funcione sin problemas \citep{Balanskat-Engelhardt}.

El pensamiento computacional generalmente es asociado con la programación , pero es mucho más que eso, implica resolver problemas, diseñar sistemas y comprender el comportamiento humano recurriendo a los conceptos fundamentales de la informática. Pensar como un informático significa más que ser capaz de programar una computadora. Requiere la capacidad de abstraer y, por lo tanto, pensar en múltiples niveles de abstracción \citep{Wing}.

La programación competitiva es considerado un deporte mental, generalmente es practicado en línea y en algunas ocasiones de forma presencial en el que a cada equipo se le otorga un solo equipo para concursar. Existen competencias en donde se puede participar de forma individual y existen otras en las que se participa en equipos, generalmente en equipos de 3 integrantes. En este tipo de concursos gana aquel equipo que resuelve la mayor cantidad de problemas planteados en el concurso, si hay equipos que tiene la misma cantidad de problemas resueltos entonces gana aquel que los resolvió en menos tiempo. 

Cada persona y equipo que participan en estos concursos de programación competitiva emplean diversos métodos y estrategias para prepararse y entrenar con el objetivo de quedar primeros o en unos de las primeras posiciones en la tabla final del ranking general del concurso para. Para ello muchos estudiantes optan por asistir a los campamentos de programación competitiva que se realizan entre 1 o 2 semanas donde entrenan intensamente el pensamiento computacional y el diseño de algoritmos.

El presente trabajo plantea mejorar el rendimiento de los participantes en los concursos de programación competitiva con formato ICPC mediante una plataforma web de entrenamiento individual y por equipos de concursantes.
