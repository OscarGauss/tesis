\subsection{PLANTEAMIENTO DEL PROBLEMA}

%Cuando se enseña en las escuelas secundarias, los profesores de otras disciplinas, como las matemáticas o la física, se hacen cargo de la informática. La misma situación ocurre en muchos otros países. Los profesores no siempre tienen la experiencia adecuada para poder enseñar ciencias de la computación. A veces incluso se les pide que enseñen ciencias de la computación, pero simplemente no saben qué hacer \citep{Combéfis-Saint_Marcq}.

\subsubsection{PROBLEMA CENTRAL}
¿Cómo incrementar el rendimiento en estudiantes para los concursos de programación competitiva del tipo ICPC? 

\subsubsection{PROBLEMAS SECUNDARIOS}
\begin{itemize}[nosep]
    \item[\textbullet] Los recursos necesarios para aprenden algoritmos, paradigmas de programación necesarios en competencias de programación competitiva debe ser seleccionado de varios plataformas y varios documentos y/o textos ocasionando que el entrenamiento sea lento.
    \item[\textbullet] La comunidad de programación competitiva en Bolivia es limitada debido a esto los estudiantes no desconocen o no se motivan para participar en estos tipo de concursos.
    \item[\textbullet] Los contenidos que se debe conocer y aprender para lograr una buena posición en estos tipos de concursos son amplios, por este motivo los equipos tienen dificultades en organizar su plan de estudio y entrenamiento.
\end{itemize}

\subsection{FORMULACIÓN DEL PROBLEMA}